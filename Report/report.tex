\documentclass[letterpaper,12pt]{article}
\usepackage[margin=1in]{geometry}
\usepackage{setspace}
\usepackage{graphicx}
\usepackage{url}


\setlength{\parindent}{0pt}
\setlength{\parskip}{.5em}
\doublespacing

\begin{document}

\begin{center}
    \par{\bf \LARGE Synthotron}
    \par{\bf \large Live Music Performance with Android}
    \par{\large 18-551 Spring 2014 Capstone Project}
    \par{\large Michael Nye (mnye), Michael Ryan (mer1)}
\end{center}


\section*{The Problem}

Music performance applications on Android are virtually non-existent. Part of
this problem is that historically, Android has not had good audio or touch
latency, and music applications require very low latency to feel responsive.
Modern Android versions have latency on the order of 100-150 ms. While this
isn't yet low enough for a traditional keyboard-like interface, it is low enough
to allow for performance aspects with an appropriate interface, with CD-quality
audio (44.1kHz, 16-bit samples) in real-time.



\begin{thebibliography}{9}
    \singlespacing
    \bibitem{caustic}
    \url{https://play.google.com/store/apps/details?id=com.singlecellsoftware.caustic}

    \bibitem{clicktrack}
    \url{https://github.com/thenyeguy/ClickTrack}

    \bibitem{opensles}
    \url{http://www.khronos.org/opensles/}

\end{thebibliography}


\section*{Final Work Breakdown}

\begin{tabular}{| c || l |}
  \hline 
  Week 1 (Feb 16 - Feb 22) & Initial proposal and presentation Feb 18 \\ 
                           & Experiment with Android audio latency Feb 22 \\ \hline
  Week 2 (Feb 23 - Mar 2)  & Nye - Begin porting ClickTrack to Android \\
                           & Ryan -  Develop barebones sequencer, piano roll, or keyboard \\ &interface on Android \\ \hline
  Week 3 (Mar 9 - Mar 15)  & Spring break!  \\ \hline
  Week 4 (Mar 16 - Mar 22) & Nye - ClickTrack functionality on Android \\ 
                           & Ryan - Combine interface with ported audio production \\ \hline
  Week 5 (Mar 23 - Mar 29) & Nye - Finish port of ClickTrack with Java interface \\
                           & Ryan - Refactor and make robust interface \\ \hline
  Week 6 (Mar 30 - Apr 5)  & \textbf{Updates} \\
                           & Begin expanding functionality for additional instruments \\ & and effects\\ \hline
  Week 7 (Apr 6 - Apr 12)  &  Continue expanding functionality \\ \hline
  Week 8 (Apr 13 - Apr 19) & Continue expanding functionality. \\
                           & Spectrum and envelope visualization tools \\ \hline
  Week 9 (Apr 20 - Apr 26) & Nye - Additional effects and cleanup of audio engine \\
                           & Ryan - Additional effects and cleanup of interface \\ \hline
  Week 10 (Apr 27 - May 3) & Completed project and working demo  \\ \hline
  Week 11 (May 4 - May 11) & Final report  \\ \hline
\end{tabular}


\end{document}
